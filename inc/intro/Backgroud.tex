% The band-pass filter design you have used in your hardware design is a simple design and does not
% have very good performance (as you have hopefully already noticed!). It is usually straightforward
% to produce a filter with a given -3 dB cut-off point, but this is not the only parameter that you need
% to consider when designing a filter.
% Figure 1 shows the low frequency amplitude response for a filter similar in design to yours. The
% rate at which the signal is attenuated (called the roll off) beneath the -3 dB cutoff (indicated by the
% dark grey lines) is 20 dB/decade; in other words when the frequency is reduced by a factor of 10,
% the attenuation increases by 20 dB. This is because the design is a first order filter.
% There are two things that are inadequate about this filter response:
% 1. There is still a reasonable amount of attenuation occurring in the range of frequencies
% we are interested in – the amplitude response in our pass band is not very flat. Only
% in the region of around 250 Hz to 400 Hz is there substantially lower than 1 dB
% attenuation.
% 2. Because the filter has a slow roll off, signals at low frequencies can still make a
% noticeable contribution to our result. At 10 Hz in this example, the input has been
% attenuated by 18 dB, or 0.126 of the original signal level. In the context of the Doppler
% radar design, people walking nearby or small movements caused by holding the radar
% unit could potentially interfere with our current filter design.
% Figure 1 - Simple BPF Amplitude Response